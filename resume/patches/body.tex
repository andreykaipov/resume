\makeatletter

% Modify the text size in the cventry so our text is consistent across sections.
\xpatchcmd{\cventry}
  {\small#7}
  {\normalsize#7}
  {}{}

% Define custom \cventry commands for a common interface across styles.

\ifboolexpr{
  test {\expandafter\ifstrequal\expandafter{\metamoderncvbody}{classic}}
} {
  \newcommand*{\cventryskill}[3][.25em]{
    \cvitem{#2}{#3}
  }

  \newcommand*{\cventryproject}[4][.25em]{
    \cvitem
      {#3}
      {#4\hfill\raggedleft\href{https://#2/#3}{\code{#2/\hintstyle{#3}}}}
  }

  \newcommand*{\cventryoneline}[7][.25em]{
    \cventry[#1]{#2}{#3}{#4}{#5}{#6}{#7}
  }
} {}

\ifboolexpr{
  test {\expandafter\ifstrequal\expandafter{\metamoderncvbody}{banking}}
  or
  test {\expandafter\ifstrequal\expandafter{\metamoderncvbody}{oldstyle}}
} {
  \newcommand*{\cventryskill}[3][.25em]{
    \begin{tabularx}
      {\textwidth}
      {>{\hsize=.17\textwidth}X X}
      \hintstyle{#2} &{#3}
    \end{tabularx}
    \par\addvspace{#1}}

  \newcommand*{\cventryproject}[4][.25em]{
    \begin{tabularx}
      {\textwidth}
      {>{\hsize=.32\textwidth}X X}
      \href{https://#2/#3}{\code{#2/\hintstyle{#3}}} & #4
    \end{tabularx}
    \par\addvspace{#1}}

  % Reminiscent of classic
  \newcommand*{\cventryoneline}[7][.25em]{
      {\bfseries#3}%
      \ifthenelse{\equal{#4}{}}{}{, {\slshape#4}}%
      \ifthenelse{\equal{#5}{}}{}{, #5}%
      \ifthenelse{\equal{#6}{}}{}{, #6}%
      .\strut%
      \hfill%
      {\slshape#2}%
    \ifx&#7&%
    \else{\\%
      \begin{minipage}{\maincolumnwidth}%
        \normalsize#7%
      \end{minipage}}\fi%
    \par\addvspace{#1}}
} {}

\newcommand*{\cventrywithcommentaligned}[4][.25em]{
  \begin{tabularx}
    {\textwidth}
    {>{\hsize=.17\textwidth}X X >{\raggedleft\hsize=.30\textwidth}X}
    \hintstyle{#2} &{#3} &{#4}
  \end{tabularx}
  \par\addvspace{#1}}

%%%
%%%
%%%

% Redefine \section for the classic body to expose the thickness of the section
% lines via \hintscolumnthickness. Can't patchcmd it. :(
% Example usage: \setlength{\hintscolumnthickness}{0.55ex}
%
\ifnum0=\pdfstrcmp{\metamoderncvbody}{classic}
  \newlength{\hintscolumnthickness}
  \setlength{\hintscolumnthickness}{0.95ex}
  % copy pasta
  \@initializelength{\baseletterheight}
  \settoheight{\baseletterheight}{\sectionstyle{o}}
  \setlength{\baseletterheight}{\baseletterheight-0.95ex}
  \RenewDocumentCommand{\section}{sm}{%
    \par\addvspace{2.5ex}%
    \phantomsection{}% reset the anchor for hyperrefs
    \addcontentsline{toc}{section}{#2}%
    \cvitem[0ex]{\strut\raggedleft\raisebox{\baseletterheight}{\color{color1}\rule{\hintscolumnwidth}{\hintscolumnthickness}}}{\strut\sectionstyle{#2}}%
    \par\nobreak\addvspace{1ex}\@afterheading}% to avoid a pagebreak after the heading
\fi

% Redefine \section for banking to expose distances between each section.
% Example usage: \setlength{\sectiondistance}{1.5ex}
\ifboolexpr{
  test {\expandafter\ifstrequal\expandafter{\metamoderncvbody}{banking}}
} {
  \newlength{\sectiondistance}
  \setlength{\sectiondistance}{2.5ex}
  \RenewDocumentCommand{\section}{sm}{%
    \par\addvspace{\sectiondistance}%
    \phantomsection{}% reset the anchor for hyperrefs
    \addcontentsline{toc}{section}{#2}%
    \if@left\else\if@fullrules\else\if@mixedrules\else%
      \sectionrule\fi\fi\fi%
    \strut\sectionstyle{#2}%
    \if@fullrules%
      \sectionrule%
    \else\if@mixedrules%
      \sectionrule%
    \else\if@right\else%
      \sectionrule\fi\fi\fi%
    \par\nobreak\addvspace{1ex}\@afterheading}
} {}

\makeatother
